\documentclass[11pt,a4paper]{article}
\usepackage{hyperref}
\begin{document}
\title{NJTI-4-EN - A four factor neuropsychological personality test in English}
\author{Christian Johansson\\
christian@neojungiantypology.com}
\date{\today}
\maketitle

\begin{abstract}
Personality-systems based on Jungs theories have received a lot of criticism over the years but is still popular. We create a test motivated by the neurobiology of personality and make interpretations of Jungs theories and explores if we can create a test that fulfills formal requirements.
Our aim is to make motivated physical reductions to biology of mental constructs related to individual differences in personality.
\end{abstract}

\section{Test construction}

\subsection{Answer-options}
We use a Likert scale with five answer-options to enable valid statistical analyses:
\begin{itemize}
  \item Strongly disagree (0)
  \item Disagree (1)
  \item Neutral (2)
  \item Agree (3)
  \item Strongly agree (4)
\end{itemize}

\subsection{Statements}
\emph{Guidelines:}
\begin{itemize}
\item All statements test only one trait
\item All statements are positive
\item Statements are formulated without any negations
\item Statements are simple
\item All statements have same weight
\end{itemize}

\subsubsection{EI factor}
From a neurobiological perspective the quantity of striatal d2 dopamine receptors manifest in the psychology of learning from negative outcomes, avoiding negative outcomes and error-related negativity in general. The striatal d2 dopamine receptors are receiving signals from the prefrontal cortex that act on go/no-go signals (approach vs avoid behavior) therefore the quantity of these receptors should cause differences in this trait. Since high and low quantity is anti-correlated, traits based on this should be anti-correlated and therefore dichotomous as well.\cite{ie1,ie2,ie3,ie4,ie5,ie6,ie7}

\paragraph{E - Doing}
\emph{Brave}
Most similar to Jungs concept of \emph{The Extraverted Type}.
`In so far it is not purely reactive to environmental stimuli, it character is constantly applicable to the actual circumstances, and it finds adequate and appropriate play within the limits of the objective situation. It has no serious tendency to transcend these bounds.`\cite[p. 418]{jung1}
Not afraid of making mistakes because mistakes are not experienced as negative as for Thinking types.
From a neuroscientific perspective Doing types should have less quantity of striatal dopamine d2 receptors compared to the Thinking types.
\begin{itemize}
  \item 12. You easily dare to take the initiative with people even though you might make a blunder
  \item 21. You can easily test things even though you might make errors
  \item 24. It's Effortless you to test your way through situations
  \item 54. It's natural for you to try activities even though you might make mistakes
  \item 67. You effortlessly dare to test your way with other people even though you might do something wrong
\end{itemize}

\paragraph{I - Thinking}
\emph{Cautious}
Most similar to Jungs concept of \emph{The Introverted Type}.
`Introverted consciousness doubtless views the external conditions, but it selects the subjective determinants as the decisive ones. The type is guided, therefore, by that factor of perception and cognition which represents the receiving subjective disposition to the sense stimulus.`\cite[p. 472]{jung1}
Wanting to act reflectively and slow. New information is excluded from this.
From a neuroscientific perspective Thinking types should have higher quantity of striatal dopamine d2 receptors compared to the Doing types.
\begin{itemize}
  \item 11. You are afraid of doing things the wrong way
  \item 22. It's natural for you to stop to try to avoid making errors
  \item 53. Your blunders affect you strongly
  \item 55. It's natural for you to refrain from activities when you think you will make a mistake
  \item 72. It's natural for you to be cautious with other people
\end{itemize}

\subsubsection{TF factor}
From a neuroscientific perspective, state-dependent connectivity to the default-mode-network (DMN) vs task-positive-network (TPN) manifests in psychological differences in the balance of social/phenomenal/moral cognition vs mechanistic/formal cognition.
Since state-dependent connectivity to the networks is anti-correlated, traits based on this should be anti-correlated and therefore dichotomous. \cite{tf1,tf2,tf3,tf4,tf5,tf6,tf7,tf8,tf9,tf10,tf11,tf12,tf13,tf14,tf15,tf16}

\paragraph{T - Technical}
\emph{Mechanical / impersonal reasoning}
`As a result of the general attitude of extraversion, thinking is orientated by the object and objective data.`\cite[p.428]{jung1}
`Its desire is to reach reality; its goal is to see how external facts fit into, and fulfill, the framework of the idea; its actual creative power is proved by the fact that this thinking can also create that idea which, though not present in the external facts, is yet the most suitable, abstract expression of them. Its task is accomplished when the idea it has fashioned seems to emerge so inevitably from the external facts that they actually prove its validity.`\cite[p.481]{jung1}
Wanting to evaluate or look at ideas, facts, decisions or conclusions from a mechanical / impersonal / analytical perspective.
From a neuroscientific perspective Technical types should have lesser state-dependent connectivity to the DMN and higher state-dependent connectivity to the TPN compared to Symbolical types.
\begin{itemize}
\item 20. You enjoy principled independent reasoning
\item 43. You like to understand things in an impersonal way
\item 57. You like doing evaluations that are impartial and according to established rules
\item 90. You enjoy evaluating things in a distanced logical way
\item 97. You enjoy ruthless matter-of-fact analyses
\end{itemize}

\paragraph{F - Symbolical}
\emph{Empathetic / sympathetic reasoning}
`It agrees with objective values.. I may feel constrained, for instance, to use the predicate ``beautiful'' or ``good'', not because I find the object `beautiful' or ``good'' from my own subjective feeling, but because it is fitting and politic so to do; and fitting it certainly is, inasmuch as a contrary opinion would disturb the general feeling situation.`\cite[page.446]{jung1}
`But a stormy emotion will be brusquely rejected with murderous coldness, unless it happens to catch the subject from the side of the unconscious, i.e.\ unless, through the animation of some primordial image, feeling is, as it were, taken captive. In which event such a woman simply feels a momentary laming, invariably producing, in due course, a still more violent resistance, which reaches the object in his most vulnerable spot. The relation to the object is, as far as possible, kept in a secure and tranquil middle state of feeling, where passion and its intemperateness are resolutely proscribed.`\cite[p. 493]{jung1}
Wanting to evaluate or look at ideas, facts, decisions or conclusions from an emotional perspective. Opinions and forms of reasoning is excluded from this.
From a neuroscientific perspective Symbolical types should have higher state-dependent connectivity to the DMN and lesser state-dependent connectivity to the TPN compared to Technological types.
\begin{itemize}
\item 36. You enjoy trying to understand an idea based on what emotions it brings up in people
\item 40. You enjoy trying to understand things based on how they affect people's mood
\item 64. You like trying to understand a conclusion based on how it makes people feel
\item 78. You like judging an idea based on how much it means to people
\item 89. You enjoy evaluating decisions based on how they make people feel
\end{itemize}

\subsubsection{SN factor}
From a neuroscientific perspective, state-dependent connectivity to the frontopolar region of the brain manifests in the psychology of counter-factual, alternative, cross-contextual, imaginary cognition. \cite{sn1,sn2,sn3,sn4,sn5,sn6,sn7,sn8,sn9,sn10,sn11,sn12,sn13,sn14,sn15,sn16,sn17,sn18,sn19,sn20,sn21,sn22,sn23}

\paragraph{S - Realist}
\emph{Obvious / present / factual}
`As sense-perception, sensation is naturally dependent upon the object.`\cite[p. 456]{jung1}
`Sensation has a preferential objective determination, and those objects which release the strongest sensation are decisive for the individual's psychology. The result of this is a pronounced sensuous hold to the object. Sensation, therefore, is a vital function, equipped with the potentest [sic] vital instinct. In so far as objects release sensations, they matter; and, in so far as it lies within the power of sensation, they are also fully accepted into consciousness, whether compatible with reasoned judgment or not.`\cite[p. 457]{jung1}
`Whereas, the extraverted sensation-type is determined by the intensity of the objective influence, the introverted type is orientated by the intensity of the subjective sensation-constituent released by the objective stimulus.`\cite[p. 501]{jung1}
Wanting to stay in context or only focusing what is given or straightforward.
From a neuroscientific perspective Realist types should have lesser state-dependent connectivity to the frontopolar region of the brain compared to the Idealist types.
\begin{itemize}
\item 13. When speculating you like to only consider alternatives which are grounded in reality
\item 31. You like to only relate things to factual experiences
\item 38. You like to base discussions only in existing things
\item 63. You enjoy to base discussions only in the realistic
\item 65. You like to consider only factual or real events
\end{itemize}

\paragraph{N - Idealist}
\emph{Hidden / alternative / hypothetical}
`Intuition as the function of unconscious perception is wholly directed upon outer objects in the extraverted attitude. Because, in the main, intuition is an unconscious process, the conscious apprehension of its nature is a very difficult matter. In consciousness, the intuitive function is represented by a certain attitude of expectation, a perceptive and penetrating vision, wherein only the subsequent result can prove, in every case, how much was\cite[p. 462]{jung1} ``perceived-into'', and how much actually lay in the object.`\cite[p. 461-462]{jung1}
`The primary function of intuition is to transmit mere images, or perceptions of relations and conditions, which could be gained by the other functions, either not at all, or only by very roundabout ways.`\cite[p. 462]{jung1}
Wanting to see beyond or between contexts and fantasize how things would be if facts were different. Events are excluded from this.
From a neuroscientific perspective Idealist types should have higher state-dependent connectivity to the frontopolar region of the brain compared to Realist types.
\begin{itemize}
\item 39. You think it's exciting to fantasize about how the world would look if the facts were different
\item 49. You think it's exciting to imagine how a theory would work in a different context
\item 56. You find searching for hypothetical similarities between different contexts to be exciting
\item 75. You enjoy fantasizing about alternative associations and the relationships between various things
\item 84. You enjoy fantasizing about possible parallels between different situations
\end{itemize}

\subsubsection{JP}
From a neuroscientific perspective the degradation speed of prefrontal dopamine and the quantity of dopamine transporters in striatum manifests in psychological differences in reward-sensitivity, task-switching costs, interference control, autonomy and collaborativeness. \cite{jp1,jp2,jp3,jp4,jp5,jp6}
Since a slow prefrontal dopamine degradation and low quantity of dopamine transporters in striatum is anti-correlated with having fast prefrontal degradation of dopamine and high level of dopamine transporters in striatum, traits based on these attributes should be anti-correlated and dichotomous as well.

\paragraph{J - Judging}
\emph{Need for control / predictability / preparation}
``The two types just depicted are almost inaccessible to external judgment. Because they are introverted and have in consequence a somewhat meagre capacity or willingness for expression, they offer but a frail handle for a telling criticism.''\cite[p. 511]{jung1}
``On the contrary, these types show very often a brusque, repelling demeanour towards the outer world, although of this they are quite unaware, and have not the least intention of showing it.''\cite[p. 511]{jung1}
``The reasonableness that characterises the conscious management of life in both these types, involves a conscious exclusion of the accidental and non-rational. Reasoning judgment, in such a psychology, represents a power that coerces the untidy and accidental things of life into definite forms; such at least is its aim. Thus, on the one hand, a definite choice is made among the possibilities of life, since only the rational choice is consciously accepted; but, on the other hand, the independence and influence of those psychic functions which perceive life's happenings are essentially restricted.''\cite[p. 454]{jung1}
``The rationality of both types is orientated objectively, and depends upon objective data. Their reasonableness corresponds with what passes as reasonable from the collective standpoint. Subjectively they consider nothing rational save what is generally considered as such. But reason is also very largely subjective and individual.''\cite[p. 455]{jung1}
Wanting to experience stability and predictability to be able to make long-term decisions or goals.
From a neuroscientific perspective Judging types should have a higher prefrontal dopamine level compared to Perceiving types.
\begin{itemize}
\item 23. When you hear about a new task you generally prefer to finish what you were previously working on before changing task
\item 61. You prefer to do one thing at a time
\item 68. You enjoy finishing what you have started before you start something new
\item 73. You like to finish what you are currently focusing on before you start focusing on something new
\item 79. You enjoy reaching closure in an activity before you do anything else
\end{itemize}

\paragraph{P - Perceiving}
\emph{Need for change / adaptation}
``From the standpoint of the rational type, the irrational might easily be represented as a rational of inferior quality; namely, when he is apprehended in the light of what happens to him. For what happens to him is not the accidental-in that he is master-but, in its stead, he is overtaken by rational judgment and rational aims. This fact is hardly comprehensible to the rational mind, but its unthinkableness merely equals the astonishment of the irrational, when he discovers someone who can set the ideas of reason above the living and actual event.''\cite[p. 469]{jung1}
``Thus, the introverted rational types unquestionably have a reasoning judgment, only it is a judgment whose leading note is subjective. The laws of logic are not necessarily deflected, since its one-sidedness lies in the premise. The premise is the predominance of the subjective factor existing beneath every conclusion and colouring every judgment.''\cite[p. 496]{jung1}
Wanting to experience changes and new impressions to adapt to new situations and events.
From a neuroscientific perspective Perceiving types should have a lower prefrontal dopamine level compared to Judging types.
\begin{itemize}
  \item 33. You enjoy to swiftly focus on something new
  \item 52. You enjoy to hastily change from one activity to another
  \item 59. You like flexible environments with rapid changes
  \item 103. You enjoy to promptly adapt to something unexpected
  \item 108. You like quick changes between tasks
\end{itemize}

\subsubsection{FAKE}
\emph{Statements that all serious testers should answer positive on.}
\begin{itemize}
  \item 62. You are person with strengths and weaknesses
  \item 66. You are a person with thoughts and feelings
  \item 69. You are a person who has multiple interests
\end{itemize}

\subsubsection{Dichotomy independence}
Individual differences in prefrontal level of dopamine and striatal dopamine transporter density is independent of: striatal dopamine d2 quantity, connectivity to frontopolar region of the brain, state-dependent connectivity to DMN and TPN.
Individual differences in connectivity to frontopolar region of the brain is independent of: state-dependent connectivity to DMN and TPN, striatal dopamine d2 receptor quantity.
Individual differences in state-dependent connectivity to DMN and TPN is independent of striatal dopamine d2 receptor quantity.
From a neuroscientific perspective this should indicate that the dichotomies should be uncorrelated.

\section{Experiment}

\subsection{Hypothesis}
A serious result on the test is a result which had activated JavaScript in their web-browser and has answered enough positive on the fake statements.
Given one experiment with 500 serious results on the online-test, Cronbach's alpha for all four factors should exceed the established accepted limit at 0.7 and the p-value for exploratory factor-analysis should be below the established level of 5% or 0.05 and we should find at least 4 orthogonal factors matching the design.

\subsection{Method}
Statements are presented in a random order. We present the 40 real statements and the 3 fake ones but do not reveal which statements are fake. We use a technology to determine if the visitor uses JavaScript as a strong indication for whether the visitor is a human since robots rarely support JavaScript. Only the results which have responded positive or neutral on the fake statements are included in the statistics.
We calculate Cronbachs alpha, extra factors through orthogonal and oblique rotation with exploratory factor analysis where the number of factors are determined using parallel analysis. We do this with experiments where each statement have been responded to at least 200 testers which fulfill the requirements stated above.
We collect data via a PHP-script and perform calculation by using established routines i R. Since we determine the number of factors to extract in exploratory factor analysis via parallel analysis the whole procedure is mechanistic and we have no part which are dependent on subjective interpretation. This means that the data goes from the test into the calculations without being affected in any

\subsection{Results}

\paragraph{2016-07-03 - 500 testers}
\begin{itemize}
\item Cronbach's alpha for factors is between 0.84 - 0.91 (SN, EI, TF, JP)
\item Cronbach's alpha for traits is between 0.69 - 0.91 (I, T, N, E, S, P, J, F)
\item Exploratory factor analysis (EFA) with promax rotation finds 4 factors where they fit perfectly with the designed factors
\item Exploratory factor analysis (EFA) with varimax rotation finds 4 factors where they fit perfectly with designed factors
\item 7 eigen-values above 1
\item p-value for varimax EFA is 9.1E-141, chi-square statistics: 1983, degrees of freedom: 626
\end{itemize}

\paragraph{2016-03-28 - 500 testers}
\begin{itemize}
\item Cronbach's alpha for factors is between 0.84 - 0.91 (TF, SN, JP, EI)
\item Cronbach's alpha for traits is between 0.77 - 0.94 (T, F, S, N, J, P, E, I)
\item Exploratory factor analysis (EFA) with promax rotation finds 5 factors where 4 fit perfectly with the designed factors
\item Exploratory factor analysis (EFA) with varimax rotation finds 5 factors where 4 of them fit perfectly with designed factors
\item eigen-values above 1
\item p-value for varimax EFA is 7.11E-92, chi-square statistics: 1583, degrees of freedom: 590
\end{itemize}

\section{Discussion}
We can conclude that the results of our experiments was according to our hypothesis. We have additionally shown that it's possible to do a interpretation of Jungs personality-types which you can find empirical evidence for and can be motivated by neurobiology.

\begin{thebibliography}{9}

\bibitem{jung1}
  Psychological Types
  C. G. Jung 1921
\bibitem{ie1}
  \href{http://www.pnas.org/content/104/41/16311.full}{Genetic triple dissociation reveals multiple roles for dopamine in reinforcement learning}
  Michael J. Frank et al., 2007
\bibitem{ie2}
  \href{http://www.sciencedirect.com/science/article/pii/S1053811914010763}{Striatal D1 and D2 signaling differentially predict learning from positive and negative outcomes}
  Sylvia M.L. Cox et al. 2015
\bibitem{ie3}
  \href{http://psycnet.apa.org/psycinfo/2002-18225-003}{The neural basis of human error processing: Reinforcement learning, dopamine, and the error-related negativity.}
  Holroyd, Clay B et al. 2002
\bibitem{ie4}
  \href{http://psycnet.apa.org/journals/rev/111/4/931/}{The Neural Basis of Error Detection: Conflict Monitoring and the Error-Related Negativity}
  Yeung, Nick et al. 2004
\bibitem{ie5}
  \href{(http://onlinelibrary.wiley.com/doi/10.1111/j.1601-183X.2012.00812.x/full}{Additive effects of the dopamine D2 receptor and dopamine transporter genes on the error-related negativity in young children}
  A. Meyer et al. 2011
\bibitem{ie6}
  \href{http://journals.lww.com/neuroreport/Abstract/2004/11150/Error_related_negativity_reflects_detection_of.27.aspx}{Error-related negativity reflects detection of negative reward prediction error}
  Yasuda, Asako et al. 2004
\bibitem{ie7}
  \href{http://www.sciencedirect.com/science/article/pii/S0306452209006551}{Genetic contributions to avoidance-based decisions: striatal D2 receptor polymorphisms}
  M.J. Frank et al. 2009
\bibitem{tf1}
  \href{http://journals.plos.org/plosone/article?id=10.1371/journal.pone.0149989}{Why Do You Believe in God? Relationships between Religious Belief, Analytic Thinking, Mentalizing and Moral Concern}
  Anthony Ian Jack et al. 2016
\bibitem{tf2}
  \href{http://www.tandfonline.com/doi/abs/10.1080/08913811.2015.975915}{Out of Touch: The Analytic Misconstrual of Social Knowledge}
  Ivelin Sardamov 2015
\bibitem{tf3}
  \href{http://journal.frontiersin.org/Journal/10.3389/fnhum.2014.00114/full}{Antagonistic neural networks underlying differentiated leadership roles}
  Richard E. Boyatwzis et al. 2014
\bibitem{tf4}
  \href{http://www.sciencedirect.com/science/article/pii/S1053811914003966}{Association between resting-state functional connectivity and empathizing/systemizing}
  Hikaru Takeuchi et al. 2014
\bibitem{tf5}
  \href{http://journals.plos.org/plosone/article?id=10.1371/journal.pone.0084782}{Regional Gray Matter Volume Is Associated with Empathising and Systemising in Young Adults}
       Hikaru Takeuchi et al. 2014
\bibitem{tf6}
  \href{http://www.sciencedirect.com/science/article/pii/S1053811912010646}{fMRI reveals reciprocal inhibition between social and physical cognitive domains}
  Anthony I. Jack et al. 2012
\bibitem{tf7}
  \href{http://www.sciencedirect.com/science/article/pii/0010028588900199}{Mental models of mechanical systems: Individual differences in qualitative and quantitative reasoning}
  Marcel Adam Just et al. 1988
\bibitem{tf8}
  \href{http://www.tandfonline.com/doi/abs/10.1080/02643290342000005}{Neural foundations for understanding social and mechanical concepts}
  Alex Martina et al. 2003
\bibitem{tf9}
  \href{http://www.sciencedirect.com/science/article/pii/S1053811913006812}{Resting state functional connectivity associated with trait emotional intelligence}
  Hikaru Takeuchi et al. 2013
\bibitem{tf10}
  \href{http://www.sciencedirect.com/science/article/pii/S0028393214002632}{Synchronous activation within the default mode network correlates with perceived social support}
  Xianwei Che et al. 2014
\bibitem{tf11}
  \href{http://cercor.oxfordjournals.org/content/early/2016/01/06/cercor.bhv325.abstract}{Processing Narratives Concerning Protected Values: A Cross-Cultural Investigation of Neural Correlates}
  Jonas T. Kaplan et al. 2016
\bibitem{tf12}
  \href{http://www.sciencedirect.com/science/article/pii/B9780123970251001731}{The Default Network and Social Cognition}
  R.N. Spreng et al. 2015
\bibitem{tf13}
  \href{http://www.mitpressjournals.org/doi/abs/10.1162/jocn.2009.21282}{Patterns of Brain Activity Supporting Autobiographical Memory, Prospection, and Theory of Mind, and Their Relationship to the Default Mode Network}
  R. Nathan Spreng et al. 2010
\bibitem{tf14}
  \href{http://jn.physiology.org/content/104/1/322.short}{Evidence for the Default Network's Role in Spontaneous Cognition}
  Jessica R. Andrews-Hanna et al. 2010
\bibitem{tf15}
  \href{http://www.sciencedirect.com/science/article/pii/S1053810008000378}{Minds at rest? Social cognition as the default mode of cognizing and its putative relationship to the 'default system' of the brain}
  Leo Schilbach et al. 2008
\bibitem{tf16}
  \href{http://www.mitpressjournals.org/doi/abs/10.1162/0898929055002418}{The Link between Social Cognition and Self-referential Thought in the Medial Prefrontal Cortex}
  Jason P. Mitchell 2005
\bibitem{sn1}
  \href{http://cercor.oxfordjournals.org/content/early/2015/01/09/cercor.bhu311.short}{Associative Recognition Memory Awareness Improved by Theta-Burst Stimulation of Frontopolar Cortex}
  Anthony J. Ryals et al. 2015
\bibitem{sn2}
  \href{http://cercor.oxfordjournals.org/content/early/2015/01/30/cercor.bhv003.short}{The Nature and Neural Correlates of Semantic Association versus Conceptual Similarity}
  Rebecca L. Jackson et al. 2015
\bibitem{sn3}
  \href{http://journals.lww.com/neuroreport/Abstract/2015/03020/Increasing_breadth_of_semantic_associations_with.13.aspx}{Increasing breadth of semantic associations with left frontopolar direct current brain stimulation: a role for individual differences}
  Brunye, Tad T. et al. 2015
\bibitem{sn4}
  \href{http://www.sciencedirect.com/science/article/pii/S0304394015002372}{The role of the frontopolar cortex in manipulation of integrated information in working memory}
  Chobok Kim et al. 2015
\bibitem{sn5}
  \href{http://www.jneurosci.org/content/35/20/7660.short}{Dynamic Network Mechanisms of Relational Integration}
  Beth L. Parkin et al. 2015
\bibitem{sn6}
  \href{http://onlinelibrary.wiley.com/doi/10.1002/hbm.22676/full}{Frontopolar activity and connectivity support dynamic conscious augmentation of creative state}
  Adam E. Green et al. 2014
\bibitem{sn7}
  \href{http://www.sciencedirect.com/science/article/pii/S0028393212001728}{Creativity and the brain: Uncovering the neural signature of conceptual expansion}
  Anna Abraham et al. 2012
\bibitem{sn8}
  \href{http://link.springer.com/article/10.3758/BF03331976}{The frontopolar cortex and human cognition}
  Kalina Christoff et al. 2013
\bibitem{sn9}
  \href{http://www.jneurosci.org/content/33/42/16657.short}{Medial and Lateral Networks in Anterior Prefrontal Cortex Support Metacognitive Ability for Memory and Perception}
  Benjamin Baird et al. 2013
\bibitem{sn10}
  \href{http://www.jneurosci.org/content/33/5/1897.short}{Anatomical Coupling between Distinct Metacognitive Systems for Memory and Visual Perception}
  Li Yan McCurdy et al. 2013
\bibitem{sn11}
  \href{http://journals.plos.org/plosbiology/article?id=10.1371/journal.pbio.1001093}{Counterfactual Choice and Learning in a Neural Network Centered on Human Lateral Frontopolar Cortex}
  Erie D. Boorman et al. 2011
\bibitem{sn12}
  \href{http://journals.plos.org/plosbiology/article?id=10.1371/journal.pbio.1001092}{Learning: Not Just the Facts, Ma'am, but the Counterfactuals as Well}
  Michael L. Platt et al. 2011
\bibitem{sn13}
  \href{http://psycnet.apa.org/journals/xlm/38/2/264/}{Neural correlates of creativity in analogical reasoning}
  Green, Adam E et al. 2011
\bibitem{sn14}
  \href{http://cercor.oxfordjournals.org/content/20/3/524.short}{Common and Dissociable Prefrontal Loci Associated with Component Mechanisms of Analogical Reasoning}
  Soohyun Ch et al. 2010
\bibitem{sn15}
  \href{http://cercor.oxfordjournals.org/content/20/1/70.short}{Connecting Long Distance: Semantic Distance in Analogical Reasoning Modulates Frontopolar Cortex Activity}
  Adam Green et al. 2009
\bibitem{sn16}
  \href{http://cercor.oxfordjournals.org/content/20/11/2647.abstract}{Specialisation of the Rostral Prefrontal Cortex for Distinct Analogy Processes}
  Emmanuelle Volle et al. 2010
\bibitem{sn17}
  \href{http://www.ncbi.nlm.nih.gov/pmc/articles/PMC2764527/pdf/nihms-130865.pdf}{The frontopolar cortex mediates event knowledge complexity: a parametric functional MRI study}
  Frank Krueger et al. 2009
\bibitem{sn18}
  \href{http://www.sciencedirect.com/science/article/pii/S089662730900395X}{Comparing the Bird in the Hand with the Ones in the Bush}
  Silvia A. Bunge et al. 2009
\bibitem{sn19}
  \href{http://www.sciencedirect.com/science/article/pii/S0896627309003894}{How Green Is the Grass on the Other Side? Frontopolar Cortex and the Evidence in Favour of Alternative Courses of Action}
  Erie D. Boorman et al. 2009
\bibitem{sn20}
  \href{http://www.jneurosci.org/content/27/14/3790.short}{Selective Retrieval of Abstract Semantic Knowledge in Left Prefrontal Cortex}
  Robert F. Goldberg et al. 2007
\bibitem{sn21}
  \href{http://www.mitpressjournals.org/doi/abs/10.1162/jocn.2007.19.8.1407}{Neural Correlates of Concreteness in Semantic Categorisation}
  Penny M. Pexman et al. 2007
\bibitem{sn22}
  \href{http://www.sciencedirect.com/science/article/pii/S0006899306010274}{Frontopolar cortex mediates abstract integration in analogy}
  Adam E. Green et al. 2006
\bibitem{sn23}
  \href{http://cercor.oxfordjournals.org/content/15/3/239}{Analogical Reasoning and Prefrontal Cortex: Evidence for Separable Retrieval and Integration Mechanisms}
  Silvia A. Bunge et al. 2005
\bibitem{sn24}
  \href{http://www.sciencedirect.com/science/article/pii/S0028393298001079}{Ventromedial prefrontal cortex mediates guessing}
  R Elliott et al. 1999
\bibitem{jp1}
  \href{http://www.ncbi.nlm.nih.gov/pmc/articles/PMC4395363/}{Reward Sensitivity Modulates Brain Activity in the Prefrontal Cortex, ACC and Striatum during Task Switching}
  Paola Fuentes-Claramonte et al. 2015
\bibitem{jp2}
  \href{http://www.sciencedirect.com/science/article/pii/S0028393210001697}{The flexible mind is associated with the catechol-O-methyltransferase (COMT) Val158Met polymorphism: Evidence for a role of dopamine in the control of task-switching}
  Lorenza S. Colzato et al. 2010
\bibitem{jp3}
  \href{http://link.springer.com/article/10.1007/s00426-013-0514-8}{Cognitive control and the COMT Val158Met polymorphism: genetic modulation of video game training and transfer to task-switching efficiency}
  Lorenza S. Colzato et al. 2013
\bibitem{jp4}
  \href{http://www.mitpressjournals.org/doi/abs/10.1162/jocn.2009.21318}{Influence of COMT Gene Polymorphism on fMRI-assessed Sustained and Transient Activity during a Working Memory Task}
  Cindy M. de Frias et al. 2010
\bibitem{jp5}
  \href{http://psycnet.apa.org/index.cfm?fa=buy.optionToBuy&uid=2008-05019-001}{Individual differences in executive functions are almost entirely genetic in origin.}
  Naomi P. et al. 2008
\bibitem{jp6}
  \href{http://www.ncbi.nlm.nih.gov/pmc/articles/PMC3111448/}{Inverted-U shaped dopamine actions on human working memory and cognitive control}
  R Cools et al. 2012
\bibitem{jp7}
  \href{http://www.nature.com/npp/journal/v35/n9/abs/npp201068a.html}{Striatal Dopamine Mediates the Interface between Motivational and Cognitive Control in Humans: Evidence from Genetic Imaging}
  Esther Aarts et al. 2010

\end{thebibliography}

\end{document}
